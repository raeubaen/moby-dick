\documentclass[12pt,a4paper]{article}
\usepackage[utf8]{inputenc}
\usepackage[italian]{babel}
\usepackage{amsmath}
\usepackage{amsfonts}
\usepackage{amssymb}
\usepackage{graphicx}
\usepackage[absolute,overlay]{textpos}
  \setlength{\TPHorizModule}{1mm}
  \setlength{\TPVertModule}{1mm}
\date{}
\title{
  Rendez-vous! - Déjà-vu? \\
  \large \textit{Crittografie Pure}}
\author{Serverus}
\begin{document}
\maketitle
\thispagestyle{empty}
  GNU nano 2.9.3                                                                                                                   Nuovo buffer                                                                                                                   Modificato  

  - Si, ma probabilmente ad orari strampalati il comitato dorme

  - Come funzionano i VGR?
  - In maniera molto flessibile e poco fiscale, perché siamo buoni, ma solo per questo precaccia.
    Ci sarà un gruppo telegram in Silent Mode a 5 minuti. Invito i capitani a contattarmi (3772622016) su Telegram così da poterli inserire.

  - Va risolto individualmente o come squadra?
  - Come squadra

  - Che succede se una squadra lo risolve prima delle altre?
  - Vince 10 minuti di anticipo sulle altre. Riceverà il primo indizio 10 minuti prima.

  - Cosa mi serve per risolvere?
  - Cultura, cervello ed esperienza. Se non sapete proprio dove mettere le mani, forse Google può aiutare.

  - Gli indizi della caccia saranno così?
  - Non direi, ma qualche crittografia ci è scappata.

\newpage
\ \\
\ \\
\begin{center}
\begin{large}
Nella piazza all'inizio di

\end{large}
\end{center}
\noindent\fbox{%
    \parbox{\textwidth}{%
  \begin{center}
        
"Strada della Collina"\\
DEL GUTALE\\
(3 3 2 6) = (3 3 8)
  \end{center}
    }%
}

\begin{center}
\begin{large}
vicino al

\end{large}\end{center}
 
\noindent\fbox{%
    \parbox{\textwidth}{%
  \begin{center}
        

"Un particolare fondo latino... Siculamente!"\\
c.l.s\\
(5 2 1) = (8)
  \end{center}
    }%
}

\begin{center}
{\large alle}
\end{center}

\noindent\fbox{%
    \parbox{\textwidth}{%
  \begin{center}
        
"Programma di approfondimento giornalistico"\\
E/2 E/2 E/2 E/2 E/2 E/2 E/2 E/2\\
(4 1 5)
  \end{center}
    }%
}


\end{document}
